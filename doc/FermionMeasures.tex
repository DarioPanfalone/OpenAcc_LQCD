

\chapter{Measurement of the chiral condensate condensate and the quark number}

The chiral condensate for a given quark is defined as 
\begin{equation}
 \langle \bar{\psi}_q \psi_q \rangle = \frac{\partial}{\partial m_q} \log Z
\end{equation}
[...]\\
On the lattice, it is calculated through noisy estimators. 
\begin{equation}
 \langle \bar{\psi}_q \psi_q \rangle  = \textrm{Tr} M^{-1} = \langle \phi 
M_q^{-1} \phi \rangle_{noise} \ , 
\end{equation}
where $M_q$ is the Dirac matrix associated to the 
quark $q$ (thus including the $SU(3)$ gauge links, which are the same for all 
quarks, and the optional $U(1)$ phase field, which may include electromagnetic 
fields and \emph{imaginary} chemical potentials), and $\langle \ldots 
\rangle_{noise}$ means an average on the noise vectors $\phi$. As a necessary 
and sufficient condition for the last equality to hold, the scalar product 
between two different noise vectors $\phi_i$ must satisfy
\begin{equation}
 \langle \phi_i \phi_j \rangle_{noise} = \delta_{ij}
\end{equation}
Given a $Z_2$ noise vector\footnote{It may also be 
gaussian, but it 
seems that $Z_2$ works best.} $\phi$, an estimate of the chiral condensate will 
be
\begin{equation}
 \langle \bar{\psi}_q \psi_q \rangle  = \phi M_q^{-1} \phi \ .
\end{equation}

The issue here is to estimate
\begin{equation}
 \label{problem}
 \chi = M_q^{-1} \phi \ . 
\end{equation}
Notice that the algorithm used in the code to invert a 
matrix (the \emph{conjugate gradient} method) requires the matrix to be 
\emph{Hermitean}. $M_q$ is not, but $M_q^\dagger M_q$ is. Then, the trick will 
be to calculate 
\begin{equation}
\label{trick}
 \chi = \left( M_q^\dagger M_q \right)^{-1} M^\dagger \phi \ . 
\end{equation}
In order to be more concrete, let's say a bit more about the structure of the 
Dirac matrix (let's forget for the sake of my fingers the subscript $q$, we're 
concerned by only a quark species at the time anyway). In the staggered 
formulation (see, for example, Rothe's textbook)
\begin{equation}
\label{dirac_matrix_u}
 M_{i,j} = \sum_{\mu = 1}^4 \left[ U_\mu(i) \delta_{i+\hat{\mu},j}- 
U_\mu^\dagger (i-\hat{\mu}) \delta_{i-\hat{\mu},j} \right] + m\delta_{i,j}
\end{equation}
where $U_\mu$ contains also \emph{all} the $U(1)$ phases you can think of 
(the staggered phases $\eta_\mu$, the imaginary chemical potentials and the 
electromagnetic fields), while $i$ and $j$ are indices for the lattice sites. 
It is very apparent that the first term only connects \emph{even} sites to 
\emph{odd} sites. Symbolically, we can write
\[ M = \left( \begin{array}{cc}
M_{ee} & M_{eo} \\
M_{oe} & M_{oo} 
\end{array}
\right) =
\left( \begin{array}{cc}
m & D_{eo} \\
D_{oe} & m 
\end{array}
\right) \ ,
\]
where the gauge links and $U(1)$ phases are contained in the $D_{eo}$ and 
$D_{oe}$ submatrices. Notice that the matrix  
\[ D = \left( \begin{array}{cc}
0 & D_{eo} \\
D_{oe} & 0 
\end{array}
\right)
\]
is anti Hermitian\footnote{There are two ways to see this: the clever one is to 
notice that $D$ is the lattice discretization of a derivative, and as such must 
be antihermitian. The \emph{poor man} way to see this (or the \emph{pedantic 
man} way to check it) is to use Eq.(\ref{dirac_matrix_u}), noticing that 
\begin{equation}
 \left(D_{eo}\right)_{ij} = - \left(D_{oe}\right)_{ji}^\dagger \ , 
\end{equation}
as it must be.
}. This means that 
\[ M^\dagger M = \left( 
\begin{array}{cc}
m^2 - D_{eo} D_{oe} & 0 \\
0 &   m^2 - D_{oe} D_{eo}
\end{array} 
\right) \ . \]
Notice that the submatrices $m^2 - D_{eo} D_{oe}$ and $m^2 - D_{oe} D_{eo}$ are 
Hermitean. So our ploy shown in Eq.(\ref{trick}) turns into
\[ \left( \begin{array}{c}
\chi_e \\
\chi_o
\end{array} \right)
= \left( 
\begin{array}{cc}
\left[m^2 - D_{eo} D_{oe}\right]^{-1} & 0 \\
0 &   \left[m^2 - D_{oe} D_{eo} \right]^{-1}
\end{array}  
\right) \left( 
\begin{array}{cc}
m   &  -D_{eo} \\
-D_{oe}   & m
\end{array}\right)
\left(
\begin{array}{c}
\phi_e \\
\phi_o
\end{array}
\right) \ ,
\] 
from which we read
\begin{equation}
\chi_e = \left[m^2 - D_{eo} D_{oe}\right]^{-1} \left( m \phi_e - D_{eo} 
\phi_o\right)
\end{equation}
and a similar equation for $\chi_o$. From this reasoning it may look like we 
need to perform two inversions, one to obtain $\chi_e$ and another to obtain 
$\chi_o$. We can avoid the second inversion by making use of the original 
statement of the problem, (Eq.~\ref{problem}), which implies $\phi = M \chi$:
\[
\left( \begin{array}{c}
\phi_e \\
\phi_o
\end{array}
\right)  =   \left( \begin{array}{cc}
m & D_{eo} \\
D_{oe} & m 
\end{array}
\right) \left( 
\begin{array}{c}
 \chi_e \\
 \chi_o
\end{array}
 \right) \ . 
\]
This tells us that $\phi_o = D_{oe} \chi_e + m \chi_o$, which implies
\[
 \chi_o = \frac{1}{m} \left( \phi_o - D_{oe} \chi_e \right) \ . 
\]
Once $\chi_e$ and $\chi_o$ are calculated, our estimate of the chiral 
condensate (for a given quark) will be
\begin{equation}
\langle \bar{\psi} \psi \rangle = \phi_e\chi_e + \phi_o \chi_o  
\end{equation}


